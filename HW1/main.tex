
%hi
\documentclass{article}
\usepackage[margin=1in]{geometry} 
\usepackage{amsmath,amsthm,amssymb,amsfonts, fancyhdr, color, comment, graphicx, environ}
\usepackage{xcolor}
\usepackage{mdframed}
\usepackage[shortlabels]{enumitem}
\usepackage{indentfirst}
\usepackage{hyperref}
\renewcommand{\footrulewidth}{0.8pt}
\hypersetup{
    colorlinks=true,
    linkcolor=blue,
    filecolor=magenta,      
    urlcolor=blue,
}


\pagestyle{fancy}



\newenvironment{problem}[2][Problem]
    { \begin{mdframed}[backgroundcolor=gray!20] \textbf{#1 #2} \\}
    {  \end{mdframed}}


\newenvironment{solution}{\textbf{Solution}}


\lhead{Kasra Amani}
\rhead{Machine Learning} 
\chead{\textbf{Assignment 1}}
\lfoot{Dr. Sharifi Zarchi}
\rfoot{Sharif University of Technology}
\def\thesection{\alph{section}}

\begin{document}
\title{\Large Machine Learning  \\[0.5cm]
        \bf\Large Assignment 1}
\author{\large Author: Kasra Amani\\ \normalsize Student No. 98101171 \\ \ \\}
\date{\large Date Last Edited: \today}

\makeatletter
    \begin{titlepage}
        \begin{center}
	   { \includegraphics[width=13cm]{sharif.png}}
	   {\ \\ \ \\}
        \vbox{}\vspace{5cm}
            {\@title }\\[3cm] 
            {\@author}
            {\large Instructor: \bf Prof. Ali Sharifi Zarchi\\ \ \\}
            {\@date\\}

        \end{center}
    \end{titlepage}
\makeatother%not necessary but looks fancy
    \begin{problem}{1}
        \begin{problem}{1.1}
                \\
                \textbf{Theorem 1:} Let $y=Ax$ where $y$ and $x$ are n-element vectors and $A$ is an $n \times n$ matrix; then: $A=\frac{\partial y}{\partial x}$.
                \\ The proof is trivial when we look at the definition of $y$.
                \\ \\ \textbf{Theorem 2:} Let the scalar $\alpha$ be as follows: $\alpha = y^T A x$ where $A$, $x$ and $y$ have similar features to theorem 1,
                we can derive that $\frac{\partial \alpha}{\partial x}= y^T A$ and $\frac{\partial \alpha}{\partial y}= x^T A^T$. \\
                Proof: $z$ (a transposition of a $n\times 1$ vector) is defined as: $z=y^T A$ (therefore $\alpha = z x$); utilizing theorem 1 we can
                derive that: $\frac{\partial\alpha}{\partial x}= z= y^T A$; also since $\alpha$ is an scalar, the following is true: $\alpha = \alpha^T= x^T A^T y$ and applying theorem
                1 again gives us: $\frac{\partial\alpha}{\partial y}= x^T A^T$.
                \\ \\ \textbf{Theorem 3:} Let $\alpha$ be of quadratic form $\alpha = x^T A x$, then: $\frac{\partial\alpha}{\partial x} = x^T(A+A^T)$.
                \\Proof: By the definition of $\alpha$ we have: $\alpha = \displaystyle\sum_{j=1}^n\sum_{i=1}^n a_{ij}x_i x_j$; differentiating according 
                to the $k$th element of \textbf{x} we get: $\frac{\partial\alpha}{\partial x_k}= \displaystyle\sum_{j=1}^n a_{kj}x_j + \sum_{i=1}^n a_{ik}x_i$ which
                holds true for all $k$ in $[1,n]$ we obtain: $\frac{\partial\alpha}{\partial x}= x^T A^T + x^T A = x^T(A^T+A)$.
            
            \begin{section}{}
                \noindent
                This theorem only holds if $A$ is symmetric; we define the scalar $\alpha$ as follows: $\alpha= x^TAx$. Then by applying theorem 3
                in a special case, we can derive $\frac{\partial\alpha}{\partial x}=2x^TA$.
            \end{section}
            \begin{section}{}
                \noindent
                Since $\alpha=x^TAx$ is a scalar, its trace is $\alpha$ itself and from there on, it is the identical theorem to theorem 3.
            \end{section}
        \end{problem}

        \begin{problem}{1.2}
            \setcounter{section}{0}
            \begin{section}{}
                \noindent
                Consider the characteristic polynomial: $p(\lambda) = |\lambda I - A| = \lambda^n + c_{n-1}\lambda^{n-1} + ... + c_1\lambda + c_0$. \\
                Since the eigenvalues are the roots of this polynomial, it can be rewritten as $(\lambda - \lambda_1)...(\lambda - \lambda_n)$. \\
                Consider $p(0)=c_0$, this can be calculated in two ways:\\
                \textbf{1. }$p(0) = (0-\lambda_1)...(0-\lambda_n) = (-1)^n\lambda_1...\lambda_n$\\
                \textbf{2. }$p(0) = |0I - 1| = |-A| = (-1)^n|A|$\\
                So we have shown that the product of eigenvalues is equal to the determinant.
            \end{section}
            \begin{section}{}
                \noindent
                Similar to the previous part, we will use two methods to calculate $c_{n-1}$;\\ \textbf{First}, we expand 
                $p(\lambda) = (\lambda - \lambda_1)...(\lambda - \lambda_n)$; this will give us $c_{n-1} = -(\lambda_1+...+\lambda_n)$.\\
                \textbf{Second}, it can be calculated by obtaining the determinant $|\lambda I-A|$:\\
                $|\lambda I-A| = \begin{vmatrix}
                    \lambda-a_{11} & -a_{12} & \dots & -a_{1n}\\
                    -a_{21} & \lambda-a_{22} &\dots & -a_{2n} \\
                    \vdots & \vdots & \ddots & \vdots \\
                    -a_{n1} & -a_{n2} & \hdots & \lambda-a_{nn}
                \end{vmatrix}
                $\\ One major part of the determinant is the product of all elements on the main diagonal ($(\lambda-a_{11})...(\lambda-a_{nn})$).
                It is clear that the remaining part has products that contain at most $n-2$ elements of the main diagonal(and therefore $n-2$ lambdas($\lambda$)).
                The second part has no effect on the value of $c_{n-1}$ and thus, this value can be calculated by finding the coefficient of $\lambda^{n-1}$
                in the first polynomial ($(\lambda-a_{11})...(\lambda-a_{nn})$) and we obtain $c_{n-1} = -(a_{11}+...+a_{nn})$.\\
                So we have $c_{n-1} = -(a_{11}+...+a_{nn}) = -(\lambda_1+...+\lambda_n)$ and since the first part is $trace(A)$, we obtain that
                if we add all eigenvalues, the result is the trace of that matrix.

            \end{section}

        \end{problem}

        \begin{problem}{1.3}
            
        \end{problem}

        \begin{problem}{1.4}
            $||A||_1 = \underset{j}{\max}(\displaystyle\sum_{i=1}^n|a_{ij}|)$\\
            $||A||_2 = \sqrt{\lambda_{max}(A^*A)}$\\
            $||A||_\infty = \underset{i}{\max}(\displaystyle\sum_{j=1}^n|a_{ij}|)$\\
            $A = \begin{bmatrix} 5&-4&2\\-1&2&3\\-2&1&0\end{bmatrix}\implies ||A||_1=8 , ||A||_2=\sqrt{51.089}=7.147 , ||A||_\infty=11$
        \end{problem}
    \end{problem} 

    \begin{problem}{2}
        \begin{problem}{2.1}

        \end{problem}

        \begin{problem}{2.2}

        \end{problem}

        \begin{problem}{2.3}
            Similar matrices: $A$ and $B$ are similar if the invertible matrix $P$ exists such that:
            $B=P^{-1}AP$ so essentially, similar matrices represent the same map under different bases where $P$ does the basis change and
            because of this, they have identical eigenvalues. Thus all we have to do, is show that $M$ and $N$ are similar where $N=P^{-1}MP$;
            consider the matrix $P^{-1}$ (which is invertible); we know that: $P^{{-1}^{-1}}NP^{-1} = PP^{-1}MPP^{-1} = M$ and we conclude that
            $M$ and $N$ are similar and thus, have the same set of eigenvalues. 
        \end{problem}
    \end{problem}

    

\end{document}